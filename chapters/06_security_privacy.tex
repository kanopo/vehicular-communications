\section{Security in Vehicular Communications}

\subsection{Applications of Inter-Vehicle Communications (IVC)}
Le Comunicazioni Inter-Veicolo vengono impiegate per una serie di funzioni critiche:
\begin{itemize}
  \item Fornire funzioni informative e di avviso.
  \item Diffondere informazioni stradali a veicoli distanti dal sito attuale. 
\end{itemize}

\subsection{Challenges in IVCs}
Le sfide associate alle IVC includono:
\begin{itemize}
  \item Gestire l'equilibrio tra responsabilità legale e privacy dei dati.
  \item Garantire comunicazioni in tempo reale per le applicazioni di assistenza alla guida. 
  \item Affrontare le sfide della scalabilità delle reti veicolari in un contesto di crescente
complessità.
\end{itemize}

\subsection{Vulnerabilities in IVCs}
Le vulnerabilità nelle comunicazioni inter-veicolo sono molteplici: 
\begin{itemize}
  \item Jamming e attacchi di tipo Denial of Service (DoS) che impediscono le
    comunicazioni. 
  \item Rischio di manomissione dei dati trasmessi tra veicoli. 
  \item Problemi di privacy e rischi di falsificazione e impersonificazione.
\end{itemize}

\subsection{Types of Attacks on IVCs}
Gli attacchi sulle IVC possono assumere diverse
forme: 
\begin{itemize}
  \item Attacchi di fabbricazione che includono la diffusione di informazioni false.
  \item Attacchi tramite tunnel e attacchi di veicoli nascosti.
  \item Manipolazione dell'identità e diffusione di informazioni ingannevoli.
\end{itemize}

\subsection{Cybersecurity Concerns}
Le preoccupazioni sulla cybersecurity nel settore automobilistico:
\begin{itemize}
  \item Aumento della sofisticatezza del cybercrime e impatto significativo sull'industria
    automobilistica.
  \item Grandi perdite finanziarie dovute a cyberattacchi come ransomware e violazioni di
    dati. 
\end{itemize}

\subsection{Security Challenges for Connected Vehicles} 
Le sfide di sicurezza per i veicoli connessi includono: 
\begin{itemize} 
  \item Sfide nell'integrazione da sensori a cloud e gestione di molteplici superfici di
    attacco. 
  \item Necessità di proteggere la privacy di conducenti e passeggeri in un mondo sempre
      più connesso.
\end{itemize}

\subsection{Examples of Hacking Incidents}
Esempi di incidenti di hacking nel settore automobilistico dal 2002 al 2010 illustrano: 
\begin{itemize} 
  \item L'evoluzione delle minacce cyber e la varietà delle tecniche di attacco
    utilizzate. 
\end{itemize}

\subsection{Cybersecurity Best Practices and Policies}
Le migliori pratiche e politiche in termini di cybersecurity includono: 
\begin{itemize} 
  \item Autovalutazione, valutazione del rischio e test di penetrazione. 
  \item Il "SPY CAR" Act, focalizzato su standard di cybersecurity e privacy. 
\end{itemize}

\subsection{Recommendations on Cybersecurity and Software Updates}
Le raccomandazioni sulla cybersecurity e gli aggiornamenti software proposti dalla UNECE
evidenziano:
\begin{itemize} 
  \item La necessità di disposizioni uniformi riguardanti la cybersecurity dei veicoli. 
\end{itemize}

\subsection{Threats for Connected Vehicles}
Le minacce per i veicoli connessi richiedono:
\begin{itemize} 
  \item Una comprensione dell'architettura di sicurezza adattiva e della comunicazione
    esterna sicura. 
  \item L'importanza di gateway sicuri nel contesto dei veicoli connessi. 
\end{itemize}

\subsection{Security Primitives for Use Cases}
Le primitive di sicurezza per diversi casi d'uso includono: 
\begin{itemize} 
  \item Autenticazione, controllo degli accessi e rilevamento delle intrusioni. 
\end{itemize}

\subsection{Cybersecurity Threats and Attack Vectors}
Le minacce alla cybersecurity e i vettori di attacco comprendono: 
\begin{itemize} 
  \item Furti di auto senza chiave, attacchi ai sistemi di infotainment, phishing e
    attacchi di forza bruta alla rete. 
  \item Rischi derivanti da dispositivi aftermarket compromessi. 
\end{itemize}

\subsection{Supply Chain Attacks and Third-party Applications}
Gli attacchi alla catena di approvvigionamento e le applicazioni di terze parti includono: 
\begin{itemize}
  \item Rischi associati alle complesse catene di approvvigionamento e all'impatto del
    ransomware. 
  \item Vulnerabilità introdotte dalle applicazioni di terze parti.
\end{itemize}

\subsection{Examples of Use Cases}
Esempi di casi d'uso nel contesto della sicurezza veicolare:  
\begin{itemize} 
  \item Importanza della privacy dei dati, integrità e autenticazione in casi come la
    fusione delle corsie e la tecnologia "vedere attraverso".
\end{itemize}

\subsection{Profiles of Attackers}
I profili degli attaccanti nel contesto della sicurezza veicolare: 
\begin{itemize} 
  \item Analisi dei diversi profili di attaccanti e dei rischi associati. 
\end{itemize}

\subsection{Open Problems in Vehicle Cybersecurity}
I problemi aperti nella cybersecurity dei veicoli includono: 
\begin{itemize} 
  \item Sfide come la verifica dei dati, il routing sicuro, la resilienza ai DoS, la
    revoca delle chiavi e l'anonimato condizionale.
\end{itemize}

\subsection{Components of a Security Architecture}
Componenti di un'architettura di sicurezza veicolare: 
\begin{itemize} 
  \item Ruoli dei registratori di dati degli eventi, dei dispositivi a prova di
    manomissione, delle PKI veicolari e dei meccanismi di autenticazione. 
  \item Considerazioni sulla privacy nel contesto della sicurezza veicolare. 
\end{itemize}

\subsection{Authenticated Location Messages and Key Management}
Messaggi di localizzazione autenticati e gestione delle chiavi: 
\begin{itemize} 
  \item Dettagli sull'autenticazione dei messaggi di sicurezza e il ruolo dei DMV nella
    gestione delle chiavi. 
\end{itemize}

\subsection{Attacks to the Infrastructures}
Attacchi alle infrastrutture: 
\begin{itemize}
  \item Esame delle vulnerabilità nelle infrastrutture, in particolare riguardo ai veicoli
    elettrici e alle stazioni di ricarica. 
\end{itemize}

