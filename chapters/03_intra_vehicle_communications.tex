\section{Intra-vehicle Communications}

\subsection{Panoramica}
Le comunicazioni intra-veicolari si concentrano su diversi elementi chiave:
\begin{itemize}
    \item Sistemi bus: fondamenti.
    \item Protocolli come K-Line, CAN, LIN, FlexRay, MOST e Ethernet in auto.
    \item Unità di Controllo Elettronico (ECU).
    \item Aspetti di sicurezza.
\end{itemize}

\subsection{Strati ISO/OSI}
L'architettura di comunicazione segue il modello a strati ISO/OSI, dove:
\begin{itemize}
    \item Ogni strato ha una funzione e un protocollo specifici.
    \item Gli strati interagiscono solo con quello immediatamente inferiore.
    \item Le interfacce seguono specifiche rigorose, spesso stabilite da enti di standardizzazione.
\end{itemize}

\subsection{Perché i Sistemi Bus?}
I sistemi bus sono preferiti per vari motivi:
\begin{itemize}
    \item Riduzione dei costi, del peso e del volume.
    \item Maggiore modularità e personalizzabilità dei veicoli.
    \item Cicli di sviluppo più brevi e maggiore riutilizzo dei componenti.
\end{itemize}

\subsection{Storia}
L'evoluzione storica comprende:
\begin{itemize}
    \item Uso dei primi microprocessori nei veicoli negli anni '80.
    \item Sviluppo del K-Line e del CAN.
    \item Introduzione dei bus di dati per la comunicazione intra-veicolare.
    \item Standardizzazione e uso in modelli di serie dal 1991.
\end{itemize}

\subsection{Casi d'Uso}
I sistemi bus come K-Line e CAN sono usati per:
\begin{itemize}
    \item Diagnosi a bordo (OBD).
    \item Controllo di motore e trasmissione.
    \item Sicurezza attiva e passiva, comfort e sistemi multimediali.
\end{itemize}

\subsection{Classificazione: Comunicazione a Bordo}
La comunicazione a bordo gestisce:
\begin{itemize}
    \item Attività complesse di controllo e monitoraggio.
    \item Trasmissioni di dati tra ECU e interfacce uomo-macchina (MMI).
    \item Sistemi bus multimediali per trasmissione di grandi volumi di dati.
\end{itemize}

\subsection{Classificazione: Comunicazione Fuori Bordo}
La comunicazione fuori bordo include:
\begin{itemize}
    \item Diagnosi e test delle emissioni di scarico.
    \item Installazione iniziale del firmware sugli ECU.
    \item Debugging durante lo sviluppo.
\end{itemize}

\subsection{Classificazione secondo il Caso d'Uso}
Diversi casi d'uso richiedono specifiche esigenze in termini di lunghezza del messaggio, frequenza, velocità dei dati e latenza.

\subsection{Topologie di Rete}
Le topologie di rete includono configurazioni lineari, stellari e ad anello, con considerazioni su costi, complessità e robustezza.

\subsection{Mezzi e Trasmissione Dati}
I mezzi di trasmissione dati includono:
\begin{itemize}
    \item Opzioni ottiche, elettriche e wireless.
    \item Modalità unicast, broadcast e multicast.
\end{itemize}

\subsection{Effetti delle Onde e Topologie di Rete}
Gli effetti delle onde sono significativi ad alti tassi di trasmissione dei dati, richiedendo contromisure come tappi terminatori e la minimizzazione dell'uso di connettori.

\subsection{Codifica dei Bit}
La codifica dei bit utilizza schemi come Non Return to Zero (NRZ) e Manchester per gestire la trasmissione dei dati.

\subsection{Riduzione dell'Interferenza Elettromagnetica (EMI)}
Per ridurre l'EMI, si possono adottare strategie come l'aggiunta di schermature ai cavi e l'uso di cablaggi a coppia intrecciata.

\subsection{Drift dell'Orologio}
Il drift dell'orologio, causato da variazioni dell'ambiente del quarzo, richiede un continuo aggiustamento della temporizzazione dei bit.

\subsection{Bit Stuffing}
Il bit stuffing è una soluzione al problema di trasmettere molti bit identici, che potrebbe lasciare senza bordi di segnale utilizzabili per compensare il drift dell'orologio.

\subsection{Classificazione secondo l'Accesso al Bus}
L'accesso al bus può essere classificato in modi diversi:
\begin{itemize}
    \item Accesso deterministico o non deterministico.
    \item Accesso centralizzato o distribuito.
    \item Con o senza rischio di collisione.
\end{itemize}

\subsection{Struttura tipica di un ECU}
Un ECU tipico è strutturato per strati, includendo il livello fisico, di accesso al bus e applicativo, spesso con un guardiano del bus per spegnimenti di emergenza.

\subsection{Punti Chiave}
Infine, i punti chiave da ricordare includono:
\begin{itemize}
    \item Topologie di rete come singolo filo e doppio filo.
    \item Differenze tra codifica Non Return to Zero (NRZ) e Manchester.
    \item Sincronizzazione per il drift dell'orologio e bit stuffing.
    \item Varie modalità di accesso al bus.
\end{itemize}
