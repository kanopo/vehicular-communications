\section{Cellular-based V2X (C-V2X)}

\subsection{Why C-V2X will be so important?}
Il C-V2X è previsto rivoluzionare i sistemi di trasporto su scala globale, garantendo
strade più sicure, miglior efficienza del traffico, riduzione delle emissioni e tempi di
viaggio, e potenziamento della mobilità personale. Combinato con la guida autonoma e la
connettività Car-to-Cloud, il C-V2X rappresenta la prossima generazione di Sistemi di
Trasporto Intelligenti (ITS) e richiede una base tecnologica di comunicazione, rete e
calcolo robusta, flessibile e agile, per cui il 5G si propone come soluzione chiave.

\subsection{V2X applications and requirements}
Le applicazioni V2X coprono un'ampia gamma di casi d'uso, raggruppati in categorie come
sicurezza, convenienza, utenti vulnerabili della strada (VRU) e assistenza avanzata alla
guida. Queste applicazioni richiedono soluzioni più performanti, che vanno oltre il
miglioramento dell'interfaccia aerea, richiedendo un approccio end-to-end più completo
come offerto dal 5G.

\subsection{Pre-C-V2X Technologies}
Le tecnologie pre-C-V2X, come IEEE 802.11p e DSRC, hanno giocato un ruolo importante nello
sviluppo delle comunicazioni veicolari. Tuttavia, hanno limitazioni significative in
termini di throughput, prestazioni in condizioni di alta densità e protezione da
interferenze e collisioni.

\subsection{C-V2X Technologies}
Il C-V2X, basato su LTE e le sue evoluzioni (come LTE Advanced e LTE Advanced Pro), è
stato standardizzato per collegare i veicoli tra loro, all'infrastruttura stradale, ad
altri utenti della strada e a servizi basati sul cloud. Offre comunicazione real-time
altamente affidabile, supportando sia trasmissioni a breve che a lungo raggio.

\subsection{Benefits of C-V2X}
I vantaggi del C-V2X includono una guida coordinata, condivisione dei sensori, avvisi per
zone di lavoro e utenti vulnerabili della strada, nonché un miglioramento della portata e
dell'affidabilità a velocità elevate del veicolo. Inoltre, offre una maggiore efficienza
spettrale, sicurezza e supporta una varietà di casi d'uso come la guida autonoma e la
raccolta di pedaggi stradali.

\subsection{C-V2X architecture and functional entities}
L'architettura del C-V2X mantiene molti blocchi funzionali dell'architettura 4G,
introducendo elementi fondamentali come il V2X Control Function e il V2X Application
Server. Copre anche le entità funzionali del C-V2X come UE, MME e le modalità di
autorizzazione del servizio C-V2X.

\subsection{Device-to-Device (D2D) for C-V2X}
Il D2D nel contesto del C-V2X supporta comunicazioni dirette tra dispositivi in prossimità
a breve raggio, offrendo vantaggi in termini di flessibilità operativa, efficienza
spettrale e riduzione dell'energia e dei costi per bit. Include una varietà di casi d'uso,
dall'offloading di dati e calcoli all'estensione della copertura.

