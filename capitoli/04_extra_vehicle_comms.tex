\section{Extra-Vehicle Car-to-X (C2X) Networking}

La suddivisione dei casi d'uso della comunicazione tra veicoli e tutto il resto (V2X) si compone
nella seguente maniera:

\begin{itemize}
	\item \textbf{NON-SAFETY}: comunicazione non legata alla sicurezza
	      \begin{itemize}
		      \item Comfort
		      \item Informazione traffico
	      \end{itemize}
	\item \textbf{SAFETY}: comunicazione legata alla sicurezza
	      \begin{itemize}
		      \item Situation awareness
		      \item Warning messages
	      \end{itemize}
\end{itemize}



\subsection{Motivazioni}

\begin{itemize}
	\item Traditional Network: \textbf{wired, non mobile, static}
	\item Mobile Ad-Hoc Network:(MANET) \textbf{wireless, mobile, dynamic}
	\item Vehicular Ad-Hoc Network: (VANET) \textbf{wireless, mobile, dynamic}
\end{itemize}

Da notare la differenza fra guida in centri urbani e guida su super strada, ci sono pro e contro
introdotti da ciascuna casistica.


\subsection{Livelli di supporto infrastrutturale}

\begin{itemize}
	\item Pure ad-hoc communication
	\item Stationary Support Unit (SSU)
	\item Road Side Unit (RSU)
	\item Traffic Information Center(TIC)
\end{itemize}

Si tende a preferire un'approccio eterogeneo per sfruttare i vantaggi di una comunicazione con e
senza infrastruttura.

\subsection{Le sfide della comunicazione C2X}



\subsubsection{Comunicazione}
\begin{itemize}
	\item Condizioni del canale fortemente variabili.
	\item Alta congestione, contesa, interferenza.
	\item Capacità del canale fortemente limitata.
\end{itemize}

\subsubsection{Networking}
\begin{itemize}
	\item Collegamenti unidirezionali.
	\item Necessità di multi-radio e multi-rete.
	\item Apparecchiature eterogenee.
\end{itemize}

\subsubsection{Mobilità}
\begin{itemize}
	\item Topologia altamente dinamica.
	\item Tuttavia, mobilità prevedibile.
	\item Ambiente eterogeneo.
\end{itemize}

\subsubsection{Sicurezza}
\begin{itemize}
	\item Nessun collegamento affidabile all'infrastruttura centrale.
	\item Garantire la privacy.
	\item Base di utenti eterogenea.
\end{itemize}


\section{Paradigmi di comunicazione}
La comunicazione wireless si divide in due macro categorie: \textbf{basata su infrastruttura} e
\textbf{non basata su infrastruttura}.

I paradigmi basati sull'utilizzo di una infrastruttura sono:
\begin{itemize}
	\item Broadcast
	\item Cellular
\end{itemize}

Per quanto riguarda la comunicazione non basata su infrastruttura abbiamo:
\begin{itemize}
	\item Short range
	\item Medium range
\end{itemize}


\subsection{Broadcast media}
\subsubsection{Traffic Message Channel (TMC)}

Gestione centralizzata delle informazioni del traffico, le informazioni provengono da vari canali
federali Americani.

Le informazioni sono codificate per avere un significato basato sulla posizione geografica, si
compongono tabelle regionali che rappresentano il codice dell'evento e il suo significato.


\subsubsection{Transport Protocol Experts Group (TPEG)}

Successore pianificato di TMC, fondato sui seguenti principi:
\begin{itemize}
	\item Espandibilità
	\item Indipendeza dal media di trasmissione
\end{itemize}


Gli obietti di TPEG sono:
\begin{itemize}
	\item Costruire il sistema alla base per \textbf{Digital Audio Broadcast (DAB)}
	\item Modulare
	\item Broadcast(mono directional) e point-to-point (multi directional)
	\item integra sicurezza mediante il CRC(Cyclic Redundancy Check)
\end{itemize}


\subsection{Cellular Networks}

Il concetto alla base dell'utilizzo della rete cellualre \`e quello di dividere il mondo in tanti
esagoni che sono le celle, ogni cella viene servita da una stazione.


L'utilizzo delle celle impone una forte gerarchia nella rete, poich\`e dopo le celle si incanalano
nella backbone del paesino, poi del paese, poi della nazione e cosi via.

\subsubsection{UMTS(3G)}

Perdita di segnale se a 290 $km/h$ si effettua una frenata brusca ma i veri problemi sono legati
alla latenza e alal capacità della rete.

Ci sono due metodologie della gestione dei canali con UMTS:
\begin{itemize}
	\item Shared channel: Random Access Channel (RACH) in uplink e il Forward Access Channel (FACH)
	      in downlink.
	\item Dedicated Transport Channel (DCH)
\end{itemize}

